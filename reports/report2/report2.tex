%%%%%%%%%%%%%%%%%%%%%%%%%%%%%%%%%%%%%%%%%%%%%%%%%%%%%%%%%%%%%%%%%%%%%%%%%%%%%%%%
%\documentclass[12pt,conference]{ieeeconf} %Github
%\documentclass[letterpaper, 12 pt, onecolumn]{ieeeconf} %Prof. Parallel

% Comment this line out
                                                          % if you need a4paper
                                                          \documentclass[a4paper, 12pt, conference]{ieeeconf}      % Use this line for a4
                                                          % paper

\IEEEoverridecommandlockouts                              % This command is only
                                                          % needed if you want to
                                                          % use the \thanks command
\overrideIEEEmargins
% See the \addtolength command later in the file to balance the column lengths
% on the last page of the document

% The following packages can be found on http:\\www.ctan.org
\usepackage{graphics} % for pdf, bitmapped graphics files
\usepackage{epsfig} % for postscript graphics files
%\usepackage{mathptmx} % assumes new font selection scheme installed
%\usepackage{times} % assumes new font selection scheme installed
\usepackage{amsmath} % assumes amsmath package installed
\usepackage{amssymb}  % assumes amsmath package installed

\usepackage{tikz}
\usetikzlibrary{shapes, arrows.meta, positioning}

\usepackage{url}
\usepackage[ruled, vlined, linesnumbered]{algorithm2e}
%\usepackage{algorithm}
\usepackage{verbatim} 
%\usepackage[noend]{algpseudocode}
\usepackage{soul, color}
\usepackage{lmodern}
\usepackage[hidelinks]{hyperref}
\usepackage{fancyhdr}
\usepackage[utf8]{inputenc}
\usepackage{fourier} 
\usepackage{array}
\usepackage{pgf}
\usepackage{makecell}
\usepackage[sorting=none]{biblatex} % For biblatex
\addbibresource{../reports.bib} % Path to your .bib file

\SetNlSty{large}{}{:}

\renewcommand\theadalign{bc}
\renewcommand\theadfont{\bfseries}
\renewcommand\theadgape{\Gape[4pt]}
\renewcommand\cellgape{\Gape[4pt]}

\newcommand{\rework}[1]{\todo[color=yellow,inline]{#1}}

\makeatletter
\newcommand{\rom}[1]{\romannumeral #1}
\newcommand{\Rom}[1]{\expandafter\@slowromancap\romannumeral #1@}
\makeatother

\pagestyle{plain} 

\title{Comparison of Network Analytics and Significance Analysis on Spotify Artist Feature Collaboration Network\\
\large Learning From Networks - Mid-term report \\}

\author{Fabio Cociancich, Luca Fantin, Alessandro Lincetto % <-this % stops a space 
\\\\ Master Degree in Computer Engineering - University of Padova \\
}

\begin{document}

\maketitle
\thispagestyle{plain}
\pagestyle{plain}



\section{Experiments}

We have determined the feasibility of performing our experiments on the CAPRI cluster \cite{capri}. The project repository can be cloned and the necessary Python packages can be installed on our own profile, making it easy to efficiently translate the code development done on our own machines into this testing environment. For any computation, we will submit our jobs to the cluster via the SLURM work scheduler. This allows us to exploit the full computational power of CAPRI while also keeping track of execution time and resource usage.

\section{Statistical hypothesis testing}

Our statistical hypothesis testing procedure will comprise of two steps. Because of the specificity of our definition of random graph, instead of analytically computing the distribution of the centrality metrics of our model, the first step will compute how similar this distribution is compared to a Gaussian distribution starting from the features computed on the actual generated graphs, through a \emph{normality test}. We will use Shapiro-Wilk test \cite{ShapiroWilk1965}, since it is considered the most powerful normality test available \cite{RazaliYap2011}. The second step will actually determine how likely it is for the features computed on the real graph to have been drawn from the same distributions as the random graph. The available tests will be determined by the output of the first step, since a lot of procedures are based on the assumption of a Gaussian population distribution.

%\addtolength{\textheight}{-5cm}   % This command serves to balance the column lengths
                                  % on the last page of the document manually. It shortens
                                  % the textheight of the last page by a suitable amount.
                                  % This command does not take effect until the next page
                                  % so it should come on the page before the last. Make
                                  % sure that you do not shorten the textheight too much.

%%%%%%%%%%%%%%%%%%%%%%%%%%%%%%%%%%%%%%%%%%%%%%%%%%%%%%%%%%%%%%%%%%%%%%%%%%%%%%%%

%%%%%%%%%%%%%%%%%%%%%%%%%%%%%%%%%%%%%%%%%%%%%%%%%%%%%%%%%%%%%%%%%%%%%%%%%%%%%%%%
\printbibliography[nottype=online]
\end{document}