%%%%%%%%%%%%%%%%%%%%%%%%%%%%%%%%%%%%%%%%%%%%%%%%%%%%%%%%%%%%%%%%%%%%%%%%%%%%%%%%
%\documentclass[12pt,conference]{ieeeconf} %Github
%\documentclass[letterpaper, 12 pt, onecolumn]{ieeeconf} %Prof. Parallel

% Comment this line out
                                                          % if you need a4paper
\documentclass[a4paper, 12pt, conference]{ieeeconf}      % Use this line for a4
                                                          % paper

\IEEEoverridecommandlockouts                              % This command is only
                                                          % needed if you want to
                                                          % use the \thanks command
\overrideIEEEmargins
% See the \addtolength command later in the file to balance the column lengths
% on the last page of the document

% The following packages can be found on http:\\www.ctan.org
\usepackage{graphics} % for pdf, bitmapped graphics files
\usepackage{epsfig} % for postscript graphics files
%\usepackage{mathptmx} % assumes new font selection scheme installed
%\usepackage{times} % assumes new font selection scheme installed
\usepackage{amsmath} % assumes amsmath package installed
\usepackage{amssymb}  % assumes amsmath package installed

\usepackage{tikz}
\usetikzlibrary{shapes, arrows.meta, positioning}

\usepackage{url}
\usepackage[ruled, vlined, linesnumbered]{algorithm2e}
%\usepackage{algorithm}
\usepackage{verbatim} 
%\usepackage[noend]{algpseudocode}
\usepackage{soul, color}
\usepackage{lmodern}
\usepackage[hidelinks]{hyperref}
\usepackage{fancyhdr}
\usepackage[utf8]{inputenc}
\usepackage{fourier} 
\usepackage{array}
\usepackage{pgf}
\usepackage{makecell}
\usepackage{subcaption}
\usepackage{csvsimple}
\usepackage{pifont}
\usepackage{booktabs}
\usepackage{caption}
\usepackage[sorting=none]{biblatex} % For biblatex
\addbibresource{../reports.bib} % Path to your .bib file

\SetNlSty{large}{}{:}

\renewcommand\theadalign{bc}
\renewcommand\theadfont{\bfseries}
\renewcommand\theadgape{\Gape[4pt]}
\renewcommand\cellgape{\Gape[4pt]}
\newcommand{\cmark}{\ding{51}}
\newcommand{\xmark}{\ding{55}}

\graphicspath{{C:/Users/ferra/OneDrive/Desktop/Uni/projects_master/lfn_project/code/lfn_project/results/rand/dist_plots}}
%\providecommand{\main}{..}
%\graphicspath{{\main/results/rand/dist_plots}}

\newcommand{\rework}[1]{\todo[color=yellow,inline]{#1}}

\makeatletter
\newcommand{\rom}[1]{\romannumeral #1}
\newcommand{\Rom}[1]{\expandafter\@slowromancap\romannumeral #1@}
\makeatother

\pagestyle{plain} 

\title{Comparison of Network Analytics and Significance Analysis on Spotify Artist Feature Collaboration Network\\
\large Learning From Networks - Final report \\}

\author{Fabio Cociancich, Luca Fantin, Alessandro Lincetto % <-this % stops a space 
\\\\ Master Degree in Computer Engineering - University of Padova \\
}

\begin{document}

\maketitle
\thispagestyle{plain}
\pagestyle{plain}

\section{Motivation}
This project analyzes the artist collaboration network on Spotify to identify musical trends. 
The goal is to study artist popularity and their connections using centrality and clustering metrics. 
 The statistical significance of the metrics calculated on the Spotify network is evaluated against random networks, for an in-depth understanding of the dynamics of artistic collaboration. 
 The analysis was performed using the NetworkX and NetworKit packages. The results provide a basis for further research.


\section{Dataset}

For this project, we used the Spotify Artist Feature Collaboration Network from Kaggle \cite{dataset}. This dataset consists of a graph where nodes correspond to artists and edges connect artists who have collaborated on at least one song. It has 156,422 nodes, which include around 20,000 artists who appeard in the Spotify weekly charts and around 136,000 artists who had at least one feature with the chart artists, and 300,386 edges between them. Out of the information included with the nodes, the ones we used to analyze our results are the following:
\begin{itemize}
    \item artist popularity, expressed as an integer number between 0 and 100 (100 corresponding to the most popular artist on the service), according to the Spotify API;
    \item list of genres, according to the Spotify API.
\end{itemize}

\section{Measures} \label{sec:measures}

The measures considered for these analyses are both graph- and node-level graph metrics. At the graph level, we compute the number of connected components and the clustering coefficient of the graph, both global and average. At the node level, instead, we have the local clustering coefficients and a series of centrality measures. Alongside those presented during the lectures (degree, closeness, betweenness, PageRank), we also considered the \emph{eigenvector centrality}, which is built on the intuition that a node is important if it is connected to other important nodes. Given a graph $G=(V,E)$, let us define $\textbf{x}\in\mathbb{R}^{|V|}$ the vector of the centrality values for all nodes in $G$, $A$ the adjacency matrix of $G$ and $\lambda\neq 0$ a constant. For any node $i$ we can write: $$x_i=\frac{1}{\lambda}\sum_{j=1}^{|V|}A_{i,j}x_j \quad \rightarrow \quad Ax=\lambda x$$
The mathematical representation of the intuition can thus be reformulated as finding the eigenvector of the adjacency matrix corresponding to the eigenvalue $\lambda$; such vector includes the values of the eigenvector centrality for all nodes. This centrality measure has been studied extensively \cite{Bonacich2007} \cite{Borgatti2006} \cite{Spizzirri2011}, also in the context of social media network analysis \cite{Maharani2014}, including Spotify \cite{South2021} \cite{South2018}.  

\section{Significance analysis framework}

The significance analysis we performed is composed by the statistical testing framework and the random graph model chosen. For the latter, we chose the \emph{Holme-Kim model} \cite{Holme2002}. The generation of a random graph starts from a number of nodes smaller than the desired graph size and no edges, iteratively adds new nodes and connects them with already existing ones with a distribution that favours nodes with an already high degree. The resulting graphs show a power-law distribution of the degrees: the probability of seeing a node with a certain degree decreases exponentially as the degree increases. Such characteristic is observed in many real-world networks and is captured by this model more accurately compared to the Erd\H{o}s-R\'{e}nyi model. Furthermore, this model produces graphs with tunable global clustering coefficients by creating additional edges: once a newly created node $v$ is connected to an existing one $w$, a new edge is created between $v$ and one of the neighbours of $w$ with a certain probability.

The null hypothesis considered in this project states that the metric values computed on the real graph well conform to the distribution determined by the generated random graphs, which implies our Spotify dataset does not have any significant feature that can explain the values we compute. Our statistical testing procedures consists of two steps. Any statistical test assumes that the considered population has a known distribution, most often a Gaussian one. Because of the specificity of our random graph model, the first step computes how similar the distribution of the metric we are considering is compared to a Gaussian distribution, through a \emph{normality test}. We used Shapiro-Wilk test \cite{ShapiroWilk1965}, since it is considered the most powerful normality test available \cite{RazaliYap2011}. The second step then checks the validity of our null hypothesis by computing the probability that the random distribution of the metric generates a value greater or lower than the real value, also called p-value.

\section{Code}

All the code developed for this project, together with results files, dataset files and more, can be found in our GitHub repository \cite{githubRepo}. The central script is \texttt{main.py}, which computes any combination of the measures presented in section \ref{sec:measures} on various graphs, depending on the command line arguments provided. The script can work on the entire dataset, subgraphs taken from the dataset with respect to certain genres or popularity thresholds, and random graphs generated with the Holme-Kim model \cite{Holme2002} with parameters specified through the command line arguments. The libraries NetworkX \cite{networkx} and NetworKit \cite{networkit} are used to represent the graphs, compute the metrics and generate the random graphs.

\section{Experimental setup}

All computations on graphs have been performed on the CAPRI High-Performance Computing (HPC) cluster \cite{capri}. The hardware capabilities and the presence of the SLURM job scheduler system allowed us to perform heavy computations in a feasible time frame. This system features the following hardware:
\begin{itemize}
    \item 16 Intel(R) Xeon(R) Gold 6130 @ 2.10GHz CPUs
    \item 6 TB DDR4 RAM
    \item 2 NVIDIA Tesla P100 16GB GPUs
    \item 40 TB of disk space
\end{itemize}

On the other side, the analysis of the data computed by the cluster has been performed on our local machines. They all employ AMD Ryzen 5/7 CPUs and RAMs ranging from 8 GBs and 24 GBs.

\section{Data analysis on whole dataset}

We started this analysis by computing all available metrics on the whole dataset. Table \ref{tab:capri} reports the execution time and RAM consumption of each metric computation. Most of them took around the same time and quantity of RAM, with some notable exceptions. Closeness and betweenness are the only centralities that require significant operations on graphs (computing distances and shortest paths), thus their execution times are considerably larger than the others. On the other hand, computing the PageRank centralities took almost twice as much memory as all other metrics. This can be expected, as NetworkX computes it via the power method, which relies heavily on matrix computations.

The centrality measures were compared between each other and against their average by ranking the artists with respect to each measure, as shown in table \ref{tab:ranking}. We can see that different centrality definitions highlight different characteristics of different nodes. In the top 5 positions for degree and PageRank centralities we can find classical composers such as Bach and Sibelius, whose works have been performed by numerous orchestras; Traditional, which is a generic tag used on Spotify to mark traditional songs coming from the folklore of any culture in the world and thus do not have a specific author; Mc Gw and Mc MN, two Brazilian producers whose many connections are a combination of other songs sampling their work and working within specific genres known online \cite{brazilianArtists}. In general, the importance given to these artists by these measures is not reflected in the real music world. Instead, closeness, betweenness and eigenvector centrality place at the top artists with a huge mainstream presence and success: rappers like such as Snoop Dogg and Gucci Mane and DJs/producers like David Guetta, Steve Aoki and Diplo. These artists are also those that have the highest average ranking across all centralities.

% TODO: citare grafico
We have also created a visual representation of the distribution of centrality measures. The values for each measure have been sorted in descending order and scaled to 1. The resulting plot shows how most centrality measures have a distribution with many nodes having a low number of collaborations and a few nodes having a very high number. Closeness centrality is the only measure with a clearly different distribution, with most values concentrated withing a narrow range, suggesting that many artists are relatively close in the network, while a small portion are significantly distant from the rest of the graph.

% TODO: citare plot
Finally, we created a similar plot for the local clustering coefficients. We see a much more irregular distribution, with almost 8,000 artists with a coefficient equal to 1. These artists are involved in every possible triangle with their neighbours, suggesting their importance in their specific neighbourhood or connected component. This hypothesis is supported by the set of popularity values for these artists to have a mean of around 23.5 and a standard deviation of around 15. Thus, these artists are generally less popular, less likely to have large neighbourhoods and more likely to be involved in a large portion of all possible triangles in their neighbourhood.

\section{Data analysis on subgraphs}
The analysis of the node-level metrics on subgraphs about several genres reveals that there are significant variability among various types of music. First, we plotted the distributions of these measures through boxplots, some of which are displayed in figure \ref{fig:boxplot}.
Betweenness shows a non-uniform distribution; some genres have concentrated values, suggesting a more homogeneous "bridge" role, while others are more dispersed. 
Similarly, closeness does not exhibit a uniform distribution, with some genres being closer to each other and others more distant, indicating heterogeneity in proximity between genres. 
The clustering coefficients generally shows high values, but with a wide distribution, highlighting that while there is a tendency to form clusters, it is not uniform. The degree centrality presents a heterogeneous distribution: some genres have more connections, while others have fewer.
The eigenvector shows a variable distribution, with some genres more concentrated at specific values. PageRank, on the other hand, displays a more concentrated distribution, with values less dispersed compared to the other metrics. In general, closeness centrality values appear to have a much wider spread across most genres compared to all other metrics.

We also computed the correlation matrix between all node-level metrics across all considered subgraphs, reported in figure \ref{fig:correlation}. It reveals that some metrics, such as betweenness, degree, PageRank, and eigenvector, are strongly correlated with each other, while the clustering coefficient appears less correlated, suggesting that the tendency to form clusters depends on distinct factors.

\section{Data analysis on random graphs}

For the significance analysis we considered the real graph and a selection of subgraphs. Because of the NetworkX implementation of the Holme-Kim, these graphs had to have more edges than nodes and a global clustering coefficient lower than 0.3 to be accurately resembled by the random graphs. For each of these graphs, 100 random graphs were generated, with the same number of nodes, a similar number of edges and a comparable global clustering coefficient. For the 

\section{Future work}

Our analysis on the whole dataset and its genre subgraphs is mainly concerned with the node-level metrics. Future extensions of our work could be comparing the graph-level metrics of these graphs to study how network dynamic change within the whole network between different genres.

Another open field is using the other features included in the dataset, like the number of followers and the number of chart hits of each artist, for the same analyses presented in this report. This also includes using the popularity level more extensively than what has been reported here.

  % Please add the following required packages to your document preamble:
  % \usepackage{booktabs}
    \begin{table*}[]
      \centering
      \begin{tabular}{|c|c|c|c|c|c|c|c|}
      \hline
      \textbf{Reference graph}                                                       & \textit{Average cc} & \textit{Global cc} & \textit{\begin{tabular}[c]{@{}c@{}}Approximate\\ global cc\end{tabular}} & \textit{\begin{tabular}[c]{@{}c@{}}Maximum\\ eigenvector\end{tabular}} & \textit{\begin{tabular}[c]{@{}c@{}}Average\\ eigenvector\end{tabular}} & \textit{\begin{tabular}[c]{@{}c@{}}Maximum\\ closeness\end{tabular}} & \textit{\begin{tabular}[c]{@{}c@{}}Average\\ closeness\end{tabular}} \\ \hline
      \textbf{House subgraph}                                                        & \xmark              & \cmark             & \cmark                                                                   & \cmark                                                                 & \cmark                                                                 & \cmark                                                               & \cmark                                                               \\ \hline
      \textbf{Pop subgraph}                                                          & \xmark              & \xmark             & \xmark                                                                   & \cmark                                                                 & \cmark                                                                 &                                                                      &                                                                      \\ \hline
      \textbf{Rap subgraph}                                                          & \xmark              & \xmark             & \xmark                                                                   & \cmark                                                                 & \cmark                                                                 & \cmark                                                               & \xmark                                                               \\ \hline
      \textbf{Whole dataset}                                                         & \xmark              & \xmark             & \xmark                                                                   & \cmark                                                                 & \cmark                                                                 &                                                                      &                                                                      \\ \hline
      \textbf{\begin{tabular}[c]{@{}c@{}}Top 10\%\\ popularity subgraph\end{tabular}} & \xmark              & \xmark             & \xmark                                                                   & \cmark                                                                 & \cmark                                                                 & \xmark                                                               & \xmark                                                               \\ \hline
      \textbf{Trap subgraph}                                                         & \xmark              & \xmark             & \xmark                                                                   & \cmark                                                                 & \cmark                                                                 & \cmark                                                               & \xmark                                                               \\ \hline
      \end{tabular}
      \caption{Results of the Shapiro-Wilk normality tests for all considered graphs. The "reference graph" is the graph to which the random graphs used in the analysis refer to. "cc" stands for "clustering coefficient".}
      \label{tab:normality}
\end{table*}
  
  \begin{table*}[]
    \centering
    \begin{tabular}{|c|c|c|c|c|c|c|c|}
    \hline
    \textbf{Reference graph}                                                        & \textit{Average cc}     & \textit{Global cc}   & \textit{\begin{tabular}[c]{@{}c@{}}Approximate\\ global cc\end{tabular}} & \textit{\begin{tabular}[c]{@{}c@{}}Maximum\\ eigenvector\end{tabular}} & \textit{\begin{tabular}[c]{@{}c@{}}Average\\ eigenvector\end{tabular}} & \textit{\begin{tabular}[c]{@{}c@{}}Maximum\\ closeness\end{tabular}} & \textit{\begin{tabular}[c]{@{}c@{}}Average\\ closeness\end{tabular}} \\ \hline
    \textbf{House subgraph}                                                         & \textbf{\textgreater{}} & \textbf{\textless{}} & \textbf{\textless{}}                                                     & \textbf{\textgreater{}}                                                & \textbf{\textgreater{}}                                                & \textbf{\textgreater{}}                                              & \textbf{\textgreater{}}                                              \\ \hline
    \textbf{Pop subgraph}                                                           & \textbf{\textgreater{}} & \textbf{\textless{}} & \textbf{\textless{}}                                                     & \textbf{\textgreater{}}                                                & \textbf{\textgreater{}}                                                & \textbf{}                                                            & \textbf{}                                                            \\ \hline
    \textbf{Rap subgraph}                                                           & \textbf{\textgreater{}} & \textbf{\textless{}} & \textbf{\textless{}}                                                     & \textbf{\textgreater{}}                                                & \textbf{\textgreater{}}                                                & \textbf{\textgreater{}}                                              & \textbf{\textgreater{}}                                              \\ \hline
    \textbf{Whole dataset}                                                          & \textbf{\textgreater{}} & \textbf{\textless{}} & \textbf{\textless{}}                                                     & \textbf{\textgreater{}}                                                & \textbf{\textless{}}                                                   & \textbf{}                                                            & \textbf{}                                                            \\ \hline
    \textbf{\begin{tabular}[c]{@{}c@{}}Top 10\%\\ popularity subgraph\end{tabular}} & \textbf{\textgreater{}} & \textbf{\textless{}} & \textbf{\textless{}}                                                     & \textbf{\textgreater{}}                                                & \textbf{\textgreater{}}                                                & \textbf{\textgreater{}}                                              & \textbf{\textgreater{}}                                              \\ \hline
    \textbf{Trap subgraph}                                                          & \textbf{\textgreater{}} & \textbf{\textless{}} & \textbf{\textless{}}                                                     & \textbf{\textgreater{}}                                                & \textbf{\textgreater{}}                                                & \textbf{\textgreater{}}                                              & \textbf{\textgreater{}}                                              \\ \hline
    \end{tabular}
    \caption{Results of the p-value computations: the contents of the cells represent how we should expect the metric values to be, compared to the values computed on the real (sub)graphs, if we were to accept our null hypothesis. "cc" stands for "clustering coefficient".}
    \label{tab:pvalue}
\end{table*}
  
  \begin{figure*}
      \centering
      \begin{subfigure}{.33\textwidth}
        \centering
        \captionsetup{justification=centering}
        \includegraphics[width=\linewidth]{dist_plot_real_avg_clustering_coeff.png}
        \caption{Whole dataset, average \\ clustering coefficient}
        \label{fig:hist1}
      \end{subfigure}%
      \begin{subfigure}{.33\textwidth}
        \centering
        \captionsetup{justification=centering}
        \includegraphics[width=\linewidth]{dist_plot_rap_global_clustering_coeff.png}
        \caption{Rap subgraph, global \\ clustering coefficient}
        \label{fig:hist2}
      \end{subfigure}
      \begin{subfigure}{.33\textwidth}
        \centering
        \captionsetup{justification=centering}
        \includegraphics[width=\linewidth]{dist_plot_top10popularity_max_closeness.png}
        \caption{Top 10\% popularity subgraph, maximum closeness}
        \label{fig:hist3}
      \end{subfigure}
      \caption{Example of histograms for the metrics computed on random graphs that do not have a Gaussian distribution according to the Shapiro-Wilk test.}
      \label{fig:hist}
  \end{figure*}
  
\subsection{Graph analysis}

\subsection{Genre subgraphs analysis}
We analyzed also some subgraphs created considering only a particular genre.
The genres with highest clustering coefficients are "latin" (0.165 avg. cc,  0.300 global cc.) 
and "trap" (0.189 avg. cc,  0.270 global cc.).

The ones with lowest clustering coefficients are "techno" (0.0017 avg. cc,  0.0029 global cc.)  
and "classical" (0.0013 avg. cc,  8.7 e-05 global cc., 1260 nodes, 775 edges, 541 
connected components).


\subsection{Popularity subgraphs analysis}



By analysing the subgraphs created considering only the 0.1\% most popular artists we can 
note that it has quite high clustering coefficients (0.277 avg. cc,  0.363 global cc.).





\section*{Contributions}

\printbibliography[nottype=online]
\end{document}
