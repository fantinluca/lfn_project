\section{Significance analysis framework}

The significance analysis we performed is composed by the statistical testing framework and the random graph model chosen. For the latter, we chose the \emph{Holme-Kim model} \cite{Holme2002}. The generation of a random graph starts from a number of nodes smaller than the desired graph size and no edges, iteratively adds new nodes and connects them with already existing ones with a distribution that favours nodes with an already high degree. The resulting graphs show a power-law distribution of the degrees: the probability of seeing a node with a certain degree decreases exponentially as the degree increases. Such characteristic is observed in many real-world networks and is captured by this model more accurately compared to the Erd\H{o}s-R\'{e}nyi model. Furthermore, this model produces graphs with tunable global clustering coefficients by creating additional edges: once a newly created node $v$ is connected to an existing one $w$, a new edge is created between $v$ and one of the neighbours of $w$ with a certain probability.

The null hypothesis considered in this project states that the metric values computed on the real graph well conform to the distribution determined by the generated random graphs, which implies our Spotify dataset does not have any significant feature that can explain the values we compute. Our statistical testing procedures consists of two steps. Any statistical test assumes that the considered population has a known distribution, most often a Gaussian one. Because of the specificity of our random graph model, the first step computes how similar the distribution of the metric we are considering is compared to a Gaussian distribution, through a \emph{normality test}. We used Shapiro-Wilk test \cite{ShapiroWilk1965}, since it is considered the most powerful normality test available \cite{RazaliYap2011}. The second step then checks the validity of our null hypothesis by computing the probability that the random distribution of the metric generates a value greater or lower than the real value, also called p-value.