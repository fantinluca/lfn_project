\section{Data analysis on whole dataset}

We started this analysis by computing all available metrics on the whole dataset. Table \ref{tab:capri} reports the execution time and RAM consumption of each metric computation. Most of them took around the same time and quantity of RAM, with some notable exceptions. Closeness and betweenness are the only centralities that require significant operations on graphs (computing distances and shortest paths), thus their execution times are considerably larger than the others. On the other hand, computing the PageRank centralities took almost twice as much memory as all other metrics. This can be expected, as NetworkX computes it via the power method, which relies heavily on matrix computations.

The centrality measures were compared between each other and against their average by ranking the artists with respect to each measure, as shown in table \ref{tab:ranking}. We can see that different centrality definitions highlight different characteristics of different nodes. In the top 5 positions for degree and PageRank centralities we can find classical composers such as Bach and Sibelius, whose works have been performed by numerous orchestras; Traditional, which is a generic tag used on Spotify to mark traditional songs coming from the folklore of any culture in the world and thus do not have a specific author; Mc Gw and Mc MN, two Brazilian producers whose many connections are a combination of other songs sampling their work and working within specific genres known online \cite{brazilianArtists}. In general, the importance given to these artists by these measures is not reflected in the real music world. Instead, closeness, betweenness and eigenvector centrality place at the top artists with a huge mainstream presence and success: rappers like such as Snoop Dogg and Gucci Mane and DJs/producers like David Guetta, Steve Aoki and Diplo. These artists are also those that have the highest average ranking across all centralities.

We have also created a visual representation of the distribution of centrality measures, represented by figure \ref{fig:centralities}. The values for each measure have been sorted in descending order and scaled to 1. The resulting plot shows how most centrality measures have a distribution with many nodes having a low number of collaborations and a few nodes having a very high number. Closeness centrality is the only measure with a clearly different distribution, with most values concentrated withing a narrow range, suggesting that many artists are relatively close in the network, while a small portion are significantly distant from the rest of the graph.

Finally, we created a similar plot for the local clustering coefficients, reported in figure \ref{fig:ccs}. We see a much more irregular distribution, with almost 8,000 artists with a coefficient equal to 1. These artists are involved in every possible triangle with their neighbours, suggesting their importance in their specific neighbourhood or connected component. If we further analyze this set of artists, we discover a mean populartiy value of around 23.5 with a standard deviation of around 15, and no genres indicated for more than 5,000 of them. Thus, these artists are generally less popular, less likely to have large neighbourhoods and more likely to be involved in all possible triangles with their neighbourhood.