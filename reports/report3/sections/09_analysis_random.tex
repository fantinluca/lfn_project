\section{Data analysis on random graphs}

For the significance analysis we considered the real graph and a selection of subgraphs. Because of the NetworkX implementation of the Holme-Kim, these graphs had to have more edges than nodes and a global clustering coefficient lower than 0.3 to be accurately resembled by the random graphs. For each of these graphs, 200 random graphs were generated, with the same number of nodes, a similar number of edges and a comparable global clustering coefficient. The last part was easy for graphs with a coefficient under 0.1, but reaching values above that threshold became almost impossible. On these generated graphs, we computed all available graph-level metrics and the maximum and average value of the closeness and eigenvector centralities, in order to represent both distributions of the node rankings with respect to centrality values. The closeness centrality was not computed for the larger graphs, corresponding to the entire dataset and the pop genre subgraph, due to time constraints.

\subsection{Normality test}

For each set of random graphs generated, all values of the metrics were tested against a Gaussian distribution using the Shapiro-Wilk test. The result, reported in table \ref{tab:normality}, was that most of the distributions could be assimilated to a Gaussian one. For the other ones, plotting the histogram of the values revealed a distribution graphically resembling a Gaussian curve. Thus, we decided to take into consideration all metrics for all sets of graphs for the next phase of statistical testing.

\subsection{p-value computation}

For each metric listed in this section, we took its value from the real (sub)graph and computed the probabilities that the Gaussian distribution corresponding to the random graphs give a value higher or lower than the real one. The results, reported in table \ref{tab:pvalue}, show that we cannot accept our null hypothesis, thus we can say that our (sub)graphs have particular features that affect the metric values in some significant way. In fact, for almost all graphs, if the real ones were generated by the same distribution as the random graphs, we should expect to see higher maximum and average values for the considered centralities, a higher average clustering coefficient and a lower global clustering coefficient. The latter results confirms the difficulties in replicating the value computed on the real graphs through the random graph model. The results for the other metrics could be due to the more fragmented nature of the real graph, which has a large number of connected components and less edges.