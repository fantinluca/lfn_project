\section{Measures} \label{sec:measures}

The measures considered for these analyses are both graph- and node-level graph metrics. At the graph level, we compute the number of connected components and the clustering coefficient of the graph, both global and average. At the node level, instead, we have the local clustering coefficients and a series of centrality measures. Alongside those presented during the lectures (degree, closeness, betweenness, PageRank), we also considered the \emph{eigenvector centrality}, which is built on the intuition that a node is important if it is connected to other important nodes. Given a graph $G=(V,E)$, let us define $\textbf{x}\in\mathbb{R}^{|V|}$ the vector of the centrality values for all nodes in $G$, $A$ the adjacency matrix of $G$ and $\lambda\neq 0$ a constant. For any node $i$ we can write: $$x_i=\frac{1}{\lambda}\sum_{j=1}^{|V|}A_{i,j}x_j \quad \rightarrow \quad Ax=\lambda x$$
The mathematical representation of the intuition can thus be reformulated as finding the eigenvector of the adjacency matrix corresponding to the eigenvalue $\lambda$; such vector includes the values of the eigenvector centrality for all nodes. This centrality measure has been studied extensively \cite{Bonacich2007} \cite{Borgatti2006} \cite{Spizzirri2011}, also in the context of social media network analysis \cite{Maharani2014}, including Spotify \cite{South2021} \cite{South2018}.  